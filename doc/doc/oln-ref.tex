\documentclass{book}
\usepackage{graphicx}
\usepackage{hevea}

\newcommand\var\textit
\newcommand\code\texttt

\begin{document}
\title{Olena Reference Manual}
\author{EPITA Reserch and Development Laboratory}
\date{FIXME:}
\maketitle

%\mbox{const image$<$I2$>$\& \var{minima_map}}, \mbox{const struct\_elt$<$E$>$\& \var{se}});X

This manual documents Olena, a generic image processing library.

Copyright 2001, 2002  Laboratoire de Recherche et D\'eveloppement de l'\'Epita.

Permission is granted to make and distribute verbatim
copies of this manual provided the copyright notice and
this permission notice are preserved on all copies.

% Permission is granted to process this file through TeX
% and print the results, provided the printed document
% carries a copying permission notice identical to this
% one except for the removal of this paragraph (this
% paragraph not being relevant to the printed manual).

Permission is granted to copy and distribute modified versions of this
manual under the conditions for verbatim copying, provided that the
entire resulting derived work is distributed under the terms of a
permission notice identical to this one.

Permission is granted to copy and distribute
translations of this manual into another language,
under the above conditions for modified versions,
except that this permission notice may be stated in a
translation approved by the Free Software Foundation.

\newpage

\chapter{Introduction}

This reference manual will eventually document any public class and
functions available in Olena.  Sadly, it only covers the morphological
processing presently.

The \texttt{demo/} directory contains a few sample programs that may be
worth looking at before digging the source or sending us an email
(\texttt{olena@lrde.epita.fr}).


\chapter{Processings}

\section{Morphological processings}

\emph{Soille} refers to \emph{P. Soille, morphological Image Analysis
-- Principals and Applications}.  Springer 1998.

\input{ref-morpho.tex}

\section{Level processings}

\input{ref-level.tex}

\end{document}
